%  $Id:  $

\documentclass[pdftex]{cmspaper}
\usepackage[bookmarksnumbered,bookmarksopen,bookmarksopenlevel=1,colorlinks,link
color=magenta,citecolor=blue,urlcolor=red,plainpages=false,pdfpagelabels]{hyperr
ef}

%
%

\begin{document}

%==============================================================================
% title page for few authors

\begin{titlepage}

% select one of the following and type in the proper number:
%   \cmsnote{2005/TBD}
%  \internalnote{2005/000}
%  \conferencereport{2005/000}
   \date{\today}

  \title{LFC evaluation as CMS Data Location Service}

%  \note{Draft Version \today...}
  \note{Draft Version 0.5}

  \begin{Authlist}
    ??? ???
       \Instfoot{wherever}{???? University, ???}
    Alessandra Fanfani
       \Instfoot{bologna}{Universit\`{a} di Bologna e Sezione dell' INFN, Bologna, ITALY}
    Stefano Belforte
       \Instfoot{trieste}{Sezione dell' INFN, Trieste, ITALY}
    Antonio Delgado, Flavia Donno, Andrea Sciaba' 
       \Instfoot{cern}{EIS, CERN}

  \end{Authlist}

%\collaboration{for the CMS collaboration}

  \begin{abstract}

    This note describes the evaluation of the LCG Local File catalog LFC as Data Location Service (DLS) component
    of the CMS Data Management (DM) project. 
\end{abstract} 

  
\end{titlepage}

\setcounter{page}{2}%JPP
\tableofcontents

%%%%%%%%%%%%%%%%%%%%%%%%%%%%%%%%%%%%%%%%%%%%%%%%%%%%%%%%%%%%%%%%%%%%%%%%%%%% 


\section{The DLS prototype based on LFC}
\label{sec:proto0}

 The Data Location Service (DLS) is part of the CMS Data Management system and allows to discover
 where replicas of the data may be located in the distributed computing system, as described in \cite{DLS}.  

 A first prototype implementing the basic DLS functionalities using LFC was developed.  

\subsection{CMS replica data structure in DLS}

The mapping of CMS index ({\em block-replica}) data structure into LFC fileds is: 
\begin{itemize}
\item block identifier, blockID  (string) $<-->$ LFCs' LFN
\item location, SE (string)  $<-->$ LFC's replica table (SURL or rather host field)
\item creation timestamp  (high precision time) $<-->$ there is no correspondance
\item last update timestamp (high precision time) $<-->$ LFC's replica last access and pin time
\item ``custodial'' attribute $<-->$ f\_type field of the LFC's replica (``P'' permanent, ``V'' volatile, etc..) 
\item other attributes  $<-->$ ?? ( user defined metadata, one field of 255chars, are associated to the LFN and not to the replica)
\end{itemize}


\subsection{DLS server}
 The DLS server is a global LFC instance. 
 \begin{itemize}
 \item The host is specified with environment variable \$LFC\_HOST. 
 \item The complete path ``/grid/cms/basedir'' must be included. \$LFC\_HOME should be also usable to allow relative paths { FIXME: but does not work yet; it has to be investigated}.

 \end{itemize}

\subsection{DLS client}

 There is a client command-line interface, based on LFC python API, to perform the basic operations: 
 \begin{itemize}
   \item Add a Replica: 
\begin{flushleft}
  $\mbox{ dlslfc-add-replica $<$fileblock1$>$  $<$SEname$>$ }$
\end{flushleft}

   \item Remove a Replica:
\begin{flushleft}
 $\mbox{ dlslfc-delete-replica $<$fileblock1$>$ $<$SEname$>$ }$
\end{flushleft}

  \item Get location (SEs) hosting a file-block:
\begin{flushleft}
$\mbox{ dlslfc-get-se --datablock $<$fileblock1$>$ }$
\end{flushleft}
  \item Get file-block on a SE:
\begin{flushleft}
$\mbox{ dlslfc-get-datablock $<$SEname$>$}$
\end{flushleft}
   \end{itemize}


To run the LFC client, you need :
\begin{itemize}
\item Globus security, i.e. the vdt\_globus\_essentials RPM
\item the lcg-dm-common RPM
\item the LFC-client RPM
\item the LFC-interfaces RPM, if you are using the Python or Perl interface
\end{itemize}
You need to create a valid proxy.
                                                                                                        
The libraries contained in the LFC and lcg-dm-common RPMs are:
                                                                                                        
/opt/lcg/lib/liblfc.so

/opt/lcg/lib/perl/lfc.so

/opt/lcg/lib/python/\_lfc.so

/opt/lcg/lib/libCsec\_plugin\_GSI.so

/opt/lcg/lib/libCsec\_plugin\_GSI\_thread.so

/opt/lcg/lib/libCsec\_plugin\_ID.so


\subsection{attributes}
 \begin{itemize}
  \item custodial

The ``custodial'' attribute of a replica can be specified using the
f\_type field of the LFC's replica (``P'' permanent, ``V'' volatile, etc..).

The replica permanent/volatile attribute is set at creation time now,
and cannot be modified afterwards. There is currently no way to update that
attribute.
The work around is to delete the existing replica an create a new one with 
the attribute modified. 
A new method to modify this attribute could be also added, if it is a CMS requirement.

[ Not a requirement for the time being]

{\bf FIXME:} Is it possible to retrieve the value of this attribute?

%% 4.bis could other types be defined, up to this beeing an N-Byte
%% code that we use as we like ? Or is there a standard e.g. in SRM ?
%%????

  \item timestamps

The replica last update timestamp is described by the LFC's replica last access.

The replica creation timestamp doesn't exist. It could be added to the catalog if 
it is a CMS requirement.

[Not a requirement for the time being]
% Not a requirement for the time being

  \item additional attributes

The addition of new replica attributes is not foreseen, since LFC is not a metadata catalog. 
Other solutions can be found for that.                                                     

 \end{itemize}


\section{Testing of perfomances}


\subsection {Get data locations}

The typical use cases will be to get locations of one fileblock or N fileblocks (with N up to 1000??).
[The query based on a single fileblock will be particularly important for data location
performed by the Resource Broker via DLI, where fileblocks will be queries one
by one (as far as it is now).]

For a single query, most of the time is probably lost in the connection 
establishment and the authentication. 
The time spent to list the locations of a single fileblock ("time dlslfc-get-se 1fileblock") 
as a function of the N locations the fileblock is hosted at is reported in Tab.~\ref{get-se-1fileblock}.

\begin{table}[!htbp]
\begin{center}
 \begin{tabular}{|l|c|}         \hline
   N  & time \\ \hline
   1  & 0.77 s \\ \hline
   5  & 0.82 s  \\ \hline
   10 & 0.90 s \\ \hline
   20 & 1.02 s \\ \hline
  100 & 2.10 s \\ \hline
\end{tabular}
\caption {Time for listing locations of a signle fileblock, when there are N locations of that fileblock. }\label{get-se-1fileblock}
\end{center}
\end{table}

Performance improvements are foreseen by the LFC developers, introducing
the new API method (lfc\_getreplicas), for the single fileblock (1-lfn) query
with  almost constant response time for any number of replicas and even more improvements 
for the bulk N-fileblock query.


The times reported in Tab.~\ref{get-se-1fileblock} were obtained from the command line tools. 
Although they involve little else than querying the LFC, the times are slightly better 
running the interesting part from within the python interpreter (avoid of loading python within 
the time measurement), as reported in Tab.~\ref{get-se-1fileblock-test}.

\begin{table}[!htbp]
\begin{center}
 \begin{tabular}{|l|c|}         \hline
   N  & time \\ \hline
   1  & 0.58 s \\ \hline
   5  & 0.61 s  \\ \hline
   10 & 0.70 s \\ \hline
   20 & 0.83 s \\ \hline
  100 & 1.90 s \\ \hline
\end{tabular}
\caption {Time for listing locations of a signle fileblock, when N locations of that fileblock are available. Times excluding python loading}\label{get-se-1fileblock-test}
\end{center}
\end{table}

%   Deletion of an entry (and its 3 replicas): 1.3 s
%  
%   Retrieve SEs where datablock is located (3 SEs): 0.6 s
%


\subsection {Timing tests for impact of sessions/transactions in the operations}

In this section preliminary timing tests including all the LFC calls within a session (i.e. single authentication with the LFC) or a transaction, are reported.

Although these tests are not completely rigorous, they show that using sessions reduces times in 
a factor of 10 in addition (Tab.~\ref{dlslfc-add}), a factor of 3 in deletion (Tab.~\ref{dlslfc-del}) and has almost of no effect in listing replicas (Tab.~\ref{dlslfc-get-se}, Tab.~\ref{dlslfc-get-db}).
LFC listing operations are not performed within a session so the obtained results are expected. There is LFC development going on in this area and it will be corrected in future LFC versions (timescale:1/2 weeks away?).


\begin{table}[!htbp]
\begin{center}
 \begin{tabular}{|l|c|c|}         \hline
   {\bf dlslfc-add}  & N=100 entries & N=1000 entries \\ \hline
 No session          & 108 s       & 1165 s \\ \hline
 session             & 9.56 s      & 90 s \\ \hline
 transaction         & 8.5 s      & 80 s \\ \hline        
\end{tabular}
\caption {Timing for adding N replicas.(N is the number of items in the query)}\label{dlslfc-add}
\end{center}
\end{table}

\begin{table}[!htbp]
\begin{center}
 \begin{tabular}{|l|c|c|}         \hline
   {\bf dlslfc-delete}  & N=100 entries & N=1000 entries \\ \hline
 No session          & 111 s     & 1159 s \\ \hline
 session             & 34 s      & 344 s \\ \hline
 transaction         & 35 s      & 337 s \\ \hline        
\end{tabular}
\caption {Timing for deleting N replicas. }\label{dlslfc-del}
\end{center}
\end{table}


\begin{table}[!htbp]
\begin{center}
 \begin{tabular}{|l|c|c|}         \hline
   {\bf dlslfc-get-se}  & N=100 entries & N=1000 entries \\ \hline
 No session          & 30.18 s     & 288 s \\ \hline
 session             & 30 s      & 257 s \\ \hline
\end{tabular}
\caption {Timing for getting data location for N fileblocks .}\label{dlslfc-get-se}
\end{center}
\end{table}

\begin{table}[!htbp]
\begin{center}
 \begin{tabular}{|l|c|c|}         \hline
   {\bf dlslfc-get-datablock}  & N=113 replicas & N=2705 replicas \\ \hline
 No session          & 43.8 s     & 717 s \\ \hline
 session             & 41 s      & 691 s \\ \hline
\end{tabular}
\caption {Timing for listing fileblocks present at a given location. The time is reported for location having N fileblocks stored there. }\label{dlslfc-get-db}
\end{center}
\end{table}

Note that the population in the database during the testing is at least several thousands, 
altough the exact population is not know (no permissions to query all the directories).
How the overall population of the database affect the performances?


\subsection{secure versus in-secure LFC}

Running an in-secure version of LFC on a different port for read-only accesses
and sharing the same Database of the secure one seems to be technically possible.
However the LFC do not reccomend nor support having in-secure LFC catalog,
even if it is read-only. It would be experiment responsability to do that, and it 
should work against a dedicated back-end database (so no other experiments are exposed).

{\bf FIXME: } are the experiment supposed to share the same back-end database?


\subsection {Other tests}

\begin{itemize}

 \item investigate what makes adding a new replica slower wrt to the bare MySQL CMS prototype. 
      
       It took about 1h 20min to insert into LFC about 2900 entries, one at the time (with old command-line).
       It took about 7min with the DLS python/MySQL prototype.
       The difference seems to be due to the authentication, because when using sessions the 
       inserting time reduces by a factor of 10. 

 \item investigate the slowness of dlslfc-get-datablock 
       Retrieving list of fileblocks in one SE will be used not often and is outside the main workflow, so perormance of that is not very important at the moment.
       
 \item test the load and timing for concurrent query
 \item test the load and timing for concurrent insertion 
                                                                                                  
  \end{itemize} 

\subsection{ Naive model for expected DLS performance}

\begin{itemize}

\item number of {\em block-replica} in a year

  The total disk and space storage divided by the average fileblock size.

  The overall storage at Tier-0 and all Tier-1/2s foreseen in 2008 (\cite{CTDR})  is about 33PB. Assuming an average file-block size of 1TB, the
  total number of {\em block-replica} is of order $10^{4}$-$10^{5}$.

\item number of file-blocks in a year

  The total size of events divided by the average fileblock size.

  Assuming the most relevant data in term of size are RAW (1.5MB/evt),
  RECO (0.25MB/evt) and MC (~4MB/evt?) data and $1.5 \times 10^{9}$ events,
  with 2 reprocessing phases, the total size is about 10 PB.
  Assuming an average file-block size of 1TB, the
  total number of file-blocks in a year is of the order $10^{4}$.

%%% RAW event has 1.5MB , RECO event has 0.25MB
%%% Sim event has 2MB , (SimDigi?? has 1.5MB), RecSim has 0.4MB
%% $1.5 \times 10^{9}$ events $\times 2.25MB$ (RAW+3RECO) = about 3PB
%% $1.5 \times 10^{9}$ events $\times 3.9MB$ (MC) = about 6PB
%
\item number of queries for analysis

 Assuming:
 \begin{itemize}
  \item the whole data sample ($\sim 10K$ file-blocks) is analysed
        with 5 different analyses every 20 days (as mentioned for AOD in \cite{CM})
  \item each file-block ($\sim 1TB$) is analysed by 1000 jobs, i.e. each job
        analyse 1GB of data
  \item each job does a query to DLS to find the location of the file-block it needs
 \end{itemize}

 \mbox{5 analysis $\times 10K$ fileblocks $\times 1000$ queries} = \mbox{ $5 \times 10^{7}$ }

 \mbox{ 20 days } = \mbox{ $1.7 \times 10^{6}$ sec}

 the rate is about 30Hz ( order of 10-100Hz )

\end{itemize}

  
%%%%%%%%%%%%%%%%%%%%%%%%%%%%%%%%%%%%%%%%%%%%%%%%%%%%%%%%%%%%%%%%%%%%%%%%%%%%

\begin{thebibliography}{9}
  \bibitem{DLS} {\bf ????}, , {\bf The CMS Data Location Service} 
  
  \bibitem{CTDR} {\bf CERN/LHCC 2005-023}, , {\bf CMS Computing Technical Design Report}
  \bibitem{CM} {\bf CMS Note 2004/031}, C. Grandi, D. Stickland,
               L. Taylor, {\bf The CMS Computing Model}

\end{thebibliography}
 
%------------------------------------------------------------------------------
\pagebreak

\end{document}
