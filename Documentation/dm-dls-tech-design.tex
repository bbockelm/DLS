%  $Id: dm-dls-tech-design.tex,v 1.3 2006/03/05 21:03:52 afanfani Exp $

\documentclass[pdftex]{cmspaper}
\usepackage[bookmarksnumbered,bookmarksopen,bookmarksopenlevel=1,colorlinks,link
color=magenta,citecolor=blue,urlcolor=red,plainpages=false,pdfpagelabels]{hyperr
ef}

%
%

\begin{document}

%==============================================================================
% title page for few authors

\begin{titlepage}

% select one of the following and type in the proper number:
%   \cmsnote{2005/TBD}
%  \internalnote{2005/000}
%  \conferencereport{2005/000}
   \date{\today}

  \title{The CMS Data Location Service}

%  \note{Draft Version \today...}
  \note{Draft Version 0.5}

  \begin{Authlist}
    ??? ???
       \Instfoot{wherever}{???? University, ???}
    Alessandra Fanfani, Claudio Grandi
       \Instfoot{bologna}{Universit\`{a} di Bologna e Sezione dell' INFN, Bologna, ITALY}
    Giovanni Ciraolo
       \Instfoot{firenze}{Universit\`{a} di Firenze e Sezione dell' INFN, Firenze, ITALY}
    Lassi Tuura
       \Instfoot{ne}{Northeastern University, Boston, MA, USA}
    Peter Elmer
       \Instfoot{prince}{Princeton University, Princeton, NJ, USA}

  \end{Authlist}

\collaboration{for the CMS collaboration}

  \begin{abstract}

    This note describes the Data Location Service (DLS) component
    of the CMS Data Management (DM) project. 
    The DLS will allow users to discover where replicas of the 
    data may be located in the distributed computing system.
    The role of the DLS as well as the first-pass API are described.
    The DLS proptotype is also presented.
\end{abstract} 

  
\end{titlepage}

\setcounter{page}{2}%JPP
\tableofcontents

%%%%%%%%%%%%%%%%%%%%%%%%%%%%%%%%%%%%%%%%%%%%%%%%%%%%%%%%%%%%%%%%%%%%%%%%%%%% 
\section{The role of the DLS}
\label{sec:intro}

%%%% DLS design note version 0.5 should have:
%%    * First version of design note for discussion
%%    * Includes first-pass API
%%    * Includes proposed testing plan for using Grid Catalogs for DLS 
%%%%

The Data Location Service (DLS) is part of the CMS Data Management system
and provides a means to locate replicas of data in the distributed computing
system, as described in the Computing TDR~\cite{CTDR}.

%%% intro
%The CMS Data Management system will provide the infrastructure
%and the tools to manage large amounts of events that will be
%available when the detector will start collecting data.
%%% DBS
The data discovery functionality will be provided by the 
Dataset Bookkeeping System (DBS)~\cite{DBS}.
The DBS will allow to determine which data exists (regardless
of their location) and how they are organized in 
term of ``file-blocks''.
A ``file-block'' is a set of files that are likely to be 
accessed together. The files are grouped together into file-blocks 
since data replication/management will be done by file-block rather than 
for individual files.
The information available from queries to DBS will be site-independent.

A separate DLS is used to track the location of the data.
The DLS is indexed by file-block and it maps file-blocks
to storage elements (SE's) where they are located.
The DLS provide only names of sites hosting the data and not the
physical location of constituent files at the sites, or the composition
of file-blocks.

Information is entered into DLS, for sites where the data is located, 
either by the production system after creating a file-block or by 
the data transfer system after transfer. Only complete file-blocks 
are registered in the DLS instance that locate data available CMS wide.

Site manager operations may also result in modification to
the DLS, for instance in case of data loss or deletion.

Queries placed to the DLS result in a list of sites that have
a given file-block. 
In order to steer jobs where data are, the DLS is contacted to 
select the sites hosting the needed data.
This translate into an explicit requirement to the Grid Workload 
Management System (WMS) for a possible set of sites. Alternatively, 
the query to the DLS may be performed by the Grid WMS itself if 
the DLS provides a compatible interface (DLI) to the Grid WMS. 


  The DLS will also provide some means for expressing the concept that
certain replicas of data are considered {\em custodial}, i.e. that the
experiment considers that copy of the data at that site to be permanent.
Sites take custodial responsibility for copies of the data, that cannot 
be removed without insuring with the experiment that either the data 
is no longer needed or that some other site takes on the custodial 
responsibilities for the data.

  In addition, in case the underlying SE system provides data access cost
estimation (e.g. whether the file-block is normally on disk or on tape),
the general policies for blocks of data may be exposed through the DLS. 

  The DLS may be provided by suitable modification or evolution of existing
general Grid components.

\subsection{DLS interactions with other DM components}

This section (try to) describe the use of file-blocks in the 
overall context of data management, reporting/outlining the 
interactions between the DLS, the Dataset Bookeeping System (DBS) 
and the data placement and data transfer system (PHEDEX).

%Three phases are identified/reported:
%
\begin{itemize}

\item Harvesting files phase
 
 Produced files are being harvested and the file-blocks are being 
 built ("open" file-blocks). The following operations are performed:
 \begin{itemize}
  \item The file-block construction is tracked in the DBS (DBS local scope first 
 or DBS global scope) {\bf FIXME: how? who/which component is going to do that?}.
  \item The individual files are transferred into a given location. The status
        of the files transfer has to be tracked. {\bf FIXME: files transfer tracked by PHEDEX? How can PHEDEX know what to transfer if the file-block is not defined yet ?}
  \item "open" file-blocks are not known to DLS. A file-block is registered
        into the DLS when the file-block is closed and all files stored 
        in the final location.
 \end{itemize}

{\bf FIXME: Need to clarify how the production system is supposed to interact with DLS.}

\item Data transfer phase

 File-blocks are being transferred. The composition of each file-block 
 in term of files it contains is fixed ("closed" file-blocks).

 \begin{itemize}
  \item  The file-blocks composition is already defined in the DBS and 
         it's immutable.
  \item  Progress in transfering the single files within the file-blocks
         to the desired destination has to be tracked by PHEDEX.
  \item  A file-block is not registered into the DLS until the whole 
         file-block has been copied to the destination. {\bf FIXME: which are the PHEDEX agents that will write into DLS? the transfer node (Tier-1s/2s) agents?}
 \end{itemize}

\item Consistency check at a site

 A file-block is defined in the DBS ("closed" file-block) and it is 
 registered into the DLS to be at a site. 
 It should be possible to trigger (at any time) a procedure to check 
 the completeness of the file-block at the site (i.e. after partial data loss).
 This check procedure requires to have a database with the file-block 
 definition (DBS?) and the list of files availables at the sites (local catalog,
 trivial ls?).
 If the check fails it means that the file-block is nolonger complete at the site and the corresponding entry in the DLS need to be removed.

\end{itemize}

In this system only complete file-blocks are transferred and 
registered in the DLS. 

{\bf FIXME: Is there a need to handle ``open'' file-blocks in DLS? In the current system the DLS keep track of mapping between file-blocks and their location, it has no means to handle/distinguish open file-blocks.}

\subsection{Global/local DLS scopes}

A global DLS instance will know the location of data available for the 
whole CMS collaboration.
Several DLS instances with a "local" scope are foreseen, for example 
to locate data being produced within a MC production system

{\bf FIXME: Need to clarify the global/local scopes and implications}.
For example the average insert rate expected 
could be higher for a DLS local instance where managed fileblocks could have a smaller size, thus increasing their number.

\subsection{Expected DLS performances}
\label{sec:expected}
\begin{itemize}
\item Number of DLS entries CMS expects (2008)

The total disk and space storage divided by the average fileblock size.
The overall storage at Tier-0 and all Tier-1/2s foreseen in 2008 (\cite{CTDR})
is about 33PB. Assuming an average file-block size of 1TB, the
total number of {\em block-replica} is of order $10^{4}$-$10^{5}$.


Number of fileblocks/year  : $10^{3}$ - $10^{4}$

replicas for each fileblock: O(10)

Number of entries/year : $10^{4}$ - $10^{5}$

\item DLS queries rate for analysis

Assuming about 1 Milion of jobs/day :
 \begin{itemize}
 \item Worst scenario is with 1 query/job 

  DLS average queries rate          :  O(10Hz)

  foreseen peak concurrent queries : $\sim$ 30 -50  

 \item More realistic scenario with queries grouped by fileblock

  100-1000 jobs/fileblock thus 10Kfileblock/day

  DLS average queries rate :  O(0.1Hz)

%  estimated peak concurrent queries : $<$ 100 with query time $\sim$ 1sec per fileblock

The 10K fileblocks are likely grouped in a few hundred (up to 1K) task/day submitted by the various users/groups.
We can assume 1000 submissions (task), 10fileblocks each. So 1K processes/day,
each needing locations for 10 entries in DLS, i.e. 1K queries, each
lasting up to 10 seconds. More or less uniformly scattered during
10 "peak hours": 100 query/hour, 10 seconds each, duty cicle is $100*10/3600=30\%$, several tens of the 100 query will overlap.
At worst one can expect up to 100 query to pile up in the same 10 seconds.
In reality one would expect much less then 10K fileblocks to be accessed every day.
It seems difficult that a central DLS may realistically need to deal with 
more then a few tens of queries at the same time.
                                                                                
There may also be users who submits smaller tasks, that process
fractions of TB, and have a high job/file-block ratio.
Let's then add 10K submissions, 1 fileblock each, 1 second of DLS query 
for each of this 10K submissions/day. This is 10Kseconds of queries/100K seconds in a day. %Using a polinomial, 100K bins, 10K objectd, the probability of a bin having $>$100, or $>$ 50 is very small.

Finally let's note that if DLS is queried by ResourceBroker,
we will not have 100 RB's, 10 is more correct, they will queue
simultaneus requests and serialize them. Similar if DLS is
queried by submitter and submitter is some central service.
                 
Overall a realistic need to cope with less then 100 simultaneus DLS clients, 
for a total of a fraction of Hertz averaged over a day.


 \end{itemize}
\item DLS insert rate for data transfer

With a transfer data volume/year of about 10 PB, thus ~1GB/sec,
and fileblock's size of 1 TB the DLS insert rate is O(0.001Hz).

\item DLS insert rate for data production 

With an expected produced data volume per year of $\sim$ 1 PB and
fileblock's size of 1 TB the DLS insert rate is O(0.0001Hz).  

\end{itemize}

A brief summary is that the DLS should be able to cope with:
\begin{itemize}
 \item number of entries / year  : O($10^{4}$) - O($10^{5}$)
 \item average query rate:  O(0.1Hz) realistic , O(10Hz) worst
 \item query time: $\sim$1 sec per fileblock
 \item concurrent queries: up to 100 simultaneous queries  
 \item average insert rate : O(0.001Hz)
\end{itemize}


%%%%%%%%%%%%%%%%%%% rough estimate.....
%\subsection{DLS in numbers...}
%
%\begin{itemize}
%
%\item number of {\em block-replica} in a year
%
%  The total disk and space storage divided by the average fileblock size.
% 
%  The overall storage at Tier-0 and all Tier-1/2s foreseen in 2008 (\cite{CTDR})
%  is about 33PB. Assuming an average file-block size of 1TB, the
%  total number of {\em block-replica} is of order $10^{4}$-$10^{5}$.
%
%\item number of file-blocks in a year
%
%  The total size of events divided by the average fileblock size.
% 
%  Assuming the most relevant data in term of size are RAW (1.5MB/evt), 
%  RECO (0.25MB/evt) and MC (~4MB/evt?) data and $1.5 \times 10^{9}$ events,
%  with 2 reprocessing phases, the total size is about 10 PB.
%  Assuming an average file-block size of 1TB, the
%  total number of file-blocks is of the order $10^{4}$.
% 
%%% RAW event has 1.5MB , RECO event has 0.25MB
%%% Sim event has 2MB , (SimDigi?? has 1.5MB), RecSim has 0.4MB  
%% $1.5 \times 10^{9}$ events $\times 2.25MB$ (RAW+3RECO) = about 3PB 
%% $1.5 \times 10^{9}$ events $\times 3.9MB$ (MC) = about 6PB
%
%\item number of queries for analysis
%
% Assuming:
% \begin{itemize}
%  \item the whole data sample ($\sim 10K$ file-blocks) is analysed
%        with 5 different analyses every 20 days (as mentioned for AOD in \cite{CM})
%  \item each file-block ($\sim 1TB$) is analysed by 1000 jobs, i.e. each job
%        analyse 1GB of data
%  \item each job does a query to DLS to find the location of the file-block it needs
% \end{itemize}
%
% \mbox{5 analysis $\times 10K$ fileblocks $\times 1000$ queries} = \mbox{ $5 \times 10^{7}$ }
%
% \mbox{ 20 days } = \mbox{ $1.7 \times 10^{6}$ sec} 
%
% the rate is about 30Hz ( order of 10-100Hz )
%
%\end{itemize}

%%%%%%%%%%%%%%%%%%%%%%%%%%%%%%%%%%%%%%%%%%%%%%%%%%%%%%%%%%%%%%%%%%%%%%%%%%%% 
\section{The DLS API}
\label{sec:dlsapi}

The Data Location Service consists of a centralized index 
server plus a set of client libraries tools at each site hosting data. 
All the operations about replica management are left to production
and data transfer system.

The index ({\em block-replica} or {\em DLS entry} ) will be a structure containing 
information like: 
\begin{itemize}
\item block identifier, blockID  (string)
\item location, SE (string)
%FIXME \item creation timestamp  (high precision time)  {\bf FIXME: how is the creation timestamp supposed to be used?}
\item last update timestamp (high precision time)
\item few state attributes, e.g. ``custodial'', ``accessibility'' (in general strings)
%\item last query timestamp. The last query timestamp could implement a caching algorithm: if a file-block is not custodial and it is not queried since longtime it could be deleted. {\bf FIMXE: do we want that or not?}
\end{itemize}

For example a data structure like:
\begin{flushleft}
\mbox{block-replica} = \{
     \mbox{blockID: fileblock1 } ,
     \mbox{location: cmsbose.bo.infn.it} ,
%     \mbox{attribute : atributevalue },
%     \mbox{creation time:   Thu Oct 6 09:00:00} ,
     \mbox{lastupdate time: Thu Oct 6 09:00:00} ,
     \mbox{custodial:       yes} ,
     \mbox{accessibility:   disk}
 \}
\end{flushleft}



The client libraries will implement a simple API like:
\begin{itemize}
\item Begin/Commit/Rollback transactions:

       begin-transaction(read$|$write)

       commit-transaction()

       rollback-transaction()

\item Add a Replica:  

    add-replica(array$[${\em block-replica}$]$)
    
     Information are entered in the DLS by the production system at 
     the end of production, by data transfer agents or by the local 
     site manager. A file-block is registered into DLS only when complete.

\item Remove a Replica:
  
      remove-replica(array$[${\em block-replica}$]$)

      This allow to remove a {\em block-replica}, for instance by site 
      manager in case of data deletion of block loss.

\item Update Replica attributes:

      update-replica(array$[$blockID,location,array$[$key,value$]$$]$)

      where key is one of the attributes and the values of the attributes are updated
      with this API. This API can be replaced by an intelligent add query.

\item Get location (SEs) hosting a file-block:  

%      array$[$location$]$ = get-location(array$[$blockID$]$)
%       get-location(array$[$blockID$]$)
       array$[${\em block-replica}$]$ = get-location(array$[$blockID$]$)

%      return array$[$blockID$]$ or array$[${\em block-replica}$]$ ??
      This allows the Workload Management system to perform data location
      and submit jobs where the data are.

\item Get file-blocks on a given location (SE):

%      array$[$blockID$]$ = get-block(array$[$location$]$)
       array$[${\em block-replica}$]$ = get-block(array$[$location$]$)
%      return array$[$blockID$]$ or array$[${\em block-replica}$]$ ??

\item {\bf FIXME: Add query on attribute:}  array$[${\em block-replica}$]$ = query(array$[$key attribute$]$)
 

\end{itemize}

\subsection{DLS attributes}
 Which are the specification for the replica attributes?
\begin{itemize}
\item custodial
\item access cost : how the SE are exposing this information? How it is supposed to be used?
\item others?? 
\end{itemize}

%%%%%%%%%%%%%%%%%%%%%%%%%%%%%%%%%%%%%%%%%%%%%%%%%%%%%%%%%%%%%%%%%%%%%%%%%%%% 
\section{The DLS prototype 0 description}
\label{sec:proto0}

 A first prototype implementing the basic DLS functionalities was developed.  
% There is a prototype 0 implementing the basic DLS functionalities. 
 The prototype is a basic system written in python based on MySQL. 
 It consists of a central server with a MySQL database as backend and 
 a set of client libraries tools. No authentication/authorization mechanisms
 are implemented.

 The DLS server :
% DLS support up to 120 threads, support for transaction 
 \begin{flushleft}
  $\mbox{ dls-server --host $<$serverhost$>$ --host\_db $<$serverhost$>$ --user $<$dbuser$>$ --passwd $<$dbpasswd$>$ --database $<$dbname$>$ --num\_thread 2}$
 \end{flushleft}
  This start a DLS server with 2 client thread on default port 18080.

 There is a client command-line interface to perform the basic operations: 
 \begin{itemize}
   \item Add a Replica: 
\begin{flushleft}
  $\mbox{ dls-add-replica --host $<$serverhost$>$ --datablock $<$fileblock1$>$ --se $<$SEname$>$ }$
\end{flushleft}

   \item Remove a Replica:
\begin{flushleft}
 $\mbox{ dls-remove-replica --host $<$serverhost$>$ --datablock $<$fileblock1$>$ --se $<$SEname$>$ }$
\end{flushleft}

  \item Get location (SEs) hosting a file-block:
\begin{flushleft}
$\mbox{ dls-get-se --datablock $<$fileblock1$>$ --host $<$serverhost$>$}$
\end{flushleft}
  \item Get file-block on a SE:
\begin{flushleft}
$\mbox{ dls-get-datablock --se $<$SEname$>$ --host $<$serverhost$>$}$
\end{flushleft}
   \end{itemize}

   
   The prototype is intended for use in testing with the CMS workload 
   management tool for analysis (CRAB~\cite{CRAB}) versions supporting 
   configuration from DBS/DLS for data discovery and location.
   The data discovery will be performed querying the DBS, that
   should give back the file-blocks that correspond to these data.
   The file-blocks location will be done querying the DLS.
%% a detail:
%   In the context of the ``old EDM'' all the files of a 
%   given dataset-owner will be in a single file-block.

   The development of this prototype allow to gain experience on 
   how to evolve from current system based on RefDB/PubDBs
   to DBS/DLS.

   The protoptype is a placeholder for evaluation of Grid catalogues,
   such as the Local File Catalog (LFC),.... 

   The DLS CLI and python API is under specification \cite{DLSAPI}.

%%%%%%%%%%%%%%%%%%%%%%%%%%%%%%%%%%%%%%%%%%%%%%%%%%%%%%%%%%%%%%%%%%%%%%%%%%%%

\begin{thebibliography}{9}
  \bibitem{CTDR} {\bf ????}, , {\bf CMS Computing Technical Design Report}
  \bibitem{DBS} {\bf DBS design note }, M. A. Afaq et al., 
    {\bf The CMS Dataset Bookkeeping System}

  \bibitem{CM} {\bf CMS Note 2004/031}, C. Grandi, D. Stickland,
               L. Taylor, {\bf The CMS Computing Model}

  \bibitem{CRAB} {\bf CRAB - CMS Remote Analysis Builder}, project homepage
               http://cmsoc.cern.ch/cms/ccs/wm/www/Crab
  \bibitem{DLSAPI} {\bf DLS client API documentation} , project homepage 
                   https://uimon.cern.ch/twiki/bin/view/CMS/DLS

  \bibitem {NOTE000} {\bf CMS Note 2005/000},
    X.Somebody et al.,
    {\em ``CMS Note Template''}.
\end{thebibliography}
 
%------------------------------------------------------------------------------
\pagebreak

\end{document}
